\documentclass[11pt]{article}
\usepackage{times}

\sloppy
\hyphenpenalty 10000

\newcommand{\tabitem}{~~\llap{\textbullet}~~}

% for Letter size

%\setlength\topmargin{0.2cm} \setlength\oddsidemargin{-0cm}
%\setlength\textheight{22cm} \setlength\textwidth{15.8cm}
%\setlength\columnsep{0.25in}  \newlength\titlebox \setlength\titlebox{2.00in}
%\setlength\headheight{5pt}   \setlength\headsep{0pt}
%\setlength\footskip{1.0cm}
%\setlength\leftmargin{0.0in}
%\pagestyle{empty}
%%%%%%%%%%%%%%%%%%%%%%%%%%%%%%%%%%%%%%%%%%%%%%%%%%%%%%%%%%%%%%%


% for A4 size

\setlength\topmargin{0.2cm} \setlength\oddsidemargin{-0cm}
\setlength\textheight{24.7cm} \setlength\textwidth{16cm}
\setlength\columnsep{0.6cm}  \newlength\titlebox \setlength\titlebox{2.00in}
\setlength\headheight{5pt}   \setlength\headsep{0pt}
\setlength\footskip{1.0cm}
\setlength\leftmargin{0.0in}
\pagestyle{empty}
%%%%%%%%%%%%%%%%%%%%%%%%%%%%%%%%%%%%%%%%%%%%%%%%%%%%%%%%%%%%%%%

\setlength{\parindent}{0in}
\setlength{\parskip}{2ex}

\pagestyle{empty}

\begin{document}

\begin{center}
    {\bf
    %\Huge
    \LARGE
    Models for Modality Annotation, MOMA 2015 
    
Workshop Programme
}

        Queen Mary University of London\\
    London, UK

\end{center}
%\documentclass[a4paper,11pt]{article}
%\usepackage{ulem}
%\usepackage{a4wide}
%\usepackage[dvipsnames,svgnames]{xcolor}
%\usepackage[pdftex]{graphicx}
%\title{Moma2015}
%% [if lt IE 9]>
%%     <script src="//html5shiv.googlecode.com/svn/trunk/html5.js"></script>
%%     <![endif]
%
%\usepackage{hyperref}
%% commands generated by html2latex
%
%
%\begin{document}

\textbf{Tuesday, 14 April, 2015 --- morning session}

\begin{tabular} {ll}
09:30 - 09:45 & Get together, introductions, technicalities.       \\

& \\

09:45 - 10:00 & Opening.\\
 & \textit{Paola Pietrandrea and Malvina Nissim}\\

& \\

10:00 - 10:30 & Towards a Unified Approach to Modality Annotation in Portuguese. \\
 & \textit{Luciana Beatrix \'Avila, Am\'alia Mendes, and Iris Hendrickx} \\

& \\

10:30 - 11:15 & coffee break \\

& \\

11:15 - 11:45 &  A hedging annotation scheme focused on epistemic phrases for informal language.\\
& \textit{Liliana Mamani Sanchez and Carl Vogel}\\

& \\

11:45 - 12:15 & Annotating modals with GraphAnno, a configurable lightweight tool for multi-level annotation.\\
& \textit{Volker Gast, Lennart Bierkandt, and Christoph Rzymski}\\

& \\

12:15 - 14:00 &  lunch       \\

& \\

\end{tabular}

\textbf{Tuesday, 14 April, 2015 --- afternoon session}

\begin{tabular}{ll}

14:00 - 15:00 & Invited Talk\\ 
& Point of View: the Semantics of Perspective and Frame of Reference. \\
& \textit{James Pustejovsky, Brandeis University}\\

& \\

15:00 - 16:00 & Round table: Modality from different perspectives. Planned contributors:       \\
%\begin{itemize}
	& -- Ilse Depraetere \\
	&  \textit{On the distribution of necessity modals in English: towards a multifactorial analysis}\\
	& -- Elisa Ghia, Malvina Nissim, and Paola Pietrandrea  \\ 
	& \textit{The annotation of epistemic modality in spoken dialogues}\\
	& -- Sapna Negi \\
	& \textit{Automatic Detection of Subjunctive Mood for Opinion Mining Tasks}\\
%\end{itemize}         

& \\

16:00 - 16:30 & coffee break  \\

& \\

16:30-18:00 & General Discussion\\
& \textit{Joint modelling of modality annotation: state of affairs and future directions.}\\

& \\

19:00 & Dinner  \\

\end{tabular}
\end{document}



\\              for information:
\\m dot nissim at rug dot nl 
\\ paolapietrandrea at gmail dot com 
% 
%       <footer>
%         <p>This project is maintained by <a href="https://github.com/malvinanissim">malvinanissim</a></p>
%         <p><small>Hosted on GitHub Pages &mdash; Theme by <a href="https://github.com/orderedlist">orderedlist</a></small></p>
%       </footer>
%       


\end{document}
