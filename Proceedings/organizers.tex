\documentclass[11pt]{article}
\usepackage[utf8]{inputenc} 
\usepackage[T1]{fontenc} % fonts to encode unicode
\usepackage{times}

% for Letter size

%\setlength\topmargin{0.2cm} \setlength\oddsidemargin{-0cm}
%\setlength\textheight{22cm} \setlength\textwidth{15.8cm}
%\setlength\columnsep{0.25in}  \newlength\titlebox \setlength\titlebox{2.00in}
%\setlength\headheight{5pt}   \setlength\headsep{0pt}
%\setlength\footskip{1.0cm}
%\setlength\leftmargin{0.0in}
%\pagestyle{empty}
%%%%%%%%%%%%%%%%%%%%%%%%%%%%%%%%%%%%%%%%%%%%%%%%%%%%%%%%%%%%%%%%%%%%%%%%%%%


% for A4 size

\setlength\topmargin{0.2cm} \setlength\oddsidemargin{-0cm}
\setlength\textheight{24.7cm} \setlength\textwidth{16cm}
\setlength\columnsep{0.6cm}  \newlength\titlebox \setlength\titlebox{2.00in}
\setlength\headheight{5pt}   \setlength\headsep{0pt}
\setlength\footskip{1.0cm}
\setlength\leftmargin{0.0in}
\pagestyle{empty}
%%%%%%%%%%%%%%%%%%%%%%%%%%%%%%%%%%%%%%%%%%%%%%%%%%%%%%%%%%%%%%%%%%%%%%%%%%%

\setlength{\parindent}{0in}
\setlength{\parskip}{2ex}

\begin{document}

%\begin{center}
%  {\Large \bf Organizers}
%\end{center}

%\vspace*{0.5cm}

%%%%%%%%%%%%%%%%%%%%%%%%%%%%%%%%%%%%%%%%%%%%%%%%%%%%%%%%%%%%%%%%%%%%%%%%

\begin{description}

\item{\bf Workshop Chairs:} \vspace{2mm} \\
Malvina Nissim, University of Groningen\\
Paola Pietrandrea, University of Tours and CNRS LLL

%\vspace{3mm}
%\item{\bf Local organization:} \vspace{2mm} \\
%\textit{Local Chairs:} Matthew Purver, Mehrnoosh Sadrzadeh \\
%\textit{Website and Hackathon:} Dmitrijs Milajevs \\
%\textit{Facilities:} Sascha Griffiths\\
%\textit{Proceedings:} Dimitri Kartsaklis

\vspace{3mm}
\item{\bf Program Committee:}\vspace{2mm} \\
Johan van der Auwera, University of Antwerp \\
Delphine Battistelli, Paris 10 Nanterre\\
Anette Frank, Heidelberg University\\
Dylan Glynn, University of Paris 8\\
Ferdinand de Haan, Oracle Language Technology Group\\
Iris Hendrickx, University of Nijmegen\\
Caterina Mauri, University of Pavia\\
Marjorie McShane, Rensselaer Polytechnic Institute\\
Roser Morante, Vrije Universiteit Amsterdam\\
James Pustejovsky, Brandeis University\\
Andrea Sans\`{o}, University of Insubria\\
Roser Saur\'{i}, University Pompeu Fabra\\
Caroline Sporleder, Trier University\\
Veronika Vincze, University of Szeged\\


\vspace{3mm}
\item{\bf Round table contributors:}\vspace{2mm} \\
Ilse Depraetere, Universit\'{e} de Lille III\\ 
Elisa Ghia, Universit\`{a} per Stranieri di Siena\\
Paola Pietrandrea, University of Tours and CNRS LLL\\
Malvina Nissim, University of Groningen\\
Sapna Negi, National University of Ireland\\

%\vspace{3mm}
%\item{\bf Additional Reviewers:} \vspace{2mm} \\
%Gary Lastminute, Emergency Relief Lab (Switzerland)

\vspace{3mm}
\item{\bf Invited Speaker:}\vspace{2mm} \\
{James Pustejovsky, Brandeis University: \\ 
\textit{Point of View: the Semantics of Perspective and Frame of Reference.}}\vspace{0.2cm} \\
\underline{Abstract}: The notion of "perspective" is employed in language to introduce a
shift in the modal accessibility to a situation by an agent
(speaker/hearer). When annotating different kinds of linguistically
related phenomena, the specification language (markup scheme) usually
reflects the needs of the immediate task. For example, in a spatial
annotation task, such as traversing a landscape or following a path,
distinct frames of reference must be distinguished in order to
correctly interpret the language of viewpoint: e.g., to your left,
north of the castle, in front of the pub. For personal or ideological
viewpoint annotation, on the other hand, the scheme must reflect the
ability to position a "view" relative to different agents. Similar
remarks hold for temporal perspective annotation, and conceptual
perspectives. In this talk, we explore a way of representing point of
view over any domain, using a modal logic of perceptual
knowledge. Among other consequences, relative frame of reference
expressions are interpreted as the composition of relational
statements about two objects relative to the agent. We show how this
scheme can be used to annotate point of view in diverse domains. 



% Panelists

% Invited Paper

\end{description}
\end{document}